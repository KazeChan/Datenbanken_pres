
\documentclass[aspectratio=169,14pt,xcolor=dvipsnames]{beamer}
\usepackage[utf8]{inputenc}
\usepackage{enumerate}
\usepackage[ngerman]{babel}
\usepackage[T1]{fontenc}
\usepackage{lmodern}
\usepackage{blindtext}
\usepackage{pdfpages}
\usepackage[font=footnotesize]{caption}
\usepackage{fancyhdr}
\usepackage{extramarks}
\usepackage[]{hyperref}
\usepackage{array}
\usepackage{longtable}
\usepackage{xcolor}
\usepackage{colortbl}
\usepackage{graphicx}
\usepackage{amssymb}
\usepackage{eurosym}
\usepackage{pgfpages}
\usepackage{xfrac}
\usepackage{here}
\usepackage{listings}
\usepackage{makecell}
\usepackage{tikz}
% we want ER + above/below + left/right
\usetikzlibrary{er,positioning}
%\usepackage{csvsimple}
\usetheme[width=0.17\textwidth]{Hannover}
\usecolortheme{whale}
\usefonttheme{professionalfonts}


\definecolor{UPblau}{RGB}{0,45,80}
\definecolor{UPhellblau}{RGB}{0,170,255}
\definecolor{UPhelllila}{RGB}{255,170,255}
\definecolor{UPhellblau2}{RGB}{0,200,185}
\definecolor{UPgrau}{RGB}{140,120,125}
\definecolor{codegray}{rgb}{0.5,0.5,0.5}
\definecolor{backcolour}{rgb}{0.95,0.95,0.92}

\setbeamertemplate{caption}[numbered]

%definitionen für Code snippet, einbinden mit  \lstinputlisting{unitetest.js}
\lstdefinestyle{mystyle}{
    backgroundcolor=\color{backcolour},   
    numberstyle=\tiny\color{codegray},
    basicstyle=\ttfamily\footnotesize,
    breakatwhitespace=false,         
    breaklines=true,                 
    captionpos=b,                    
    keepspaces=true,                 
    numbers=left,                    
    numbersep=5pt,                  
    showspaces=false,                
    showstringspaces=false,
    showtabs=false,                  
    tabsize=2
}
\lstset{style=mystyle}

\title{Datenbanken}
\author[]{Laura Pech}
\institute{Unitedprint.com SE\\Auszubildende Fachinformatikerin für Anwendungsentwicklung}
\date{\today}

\addtobeamertemplate{navigation symbols}{}{%
    \usebeamerfont{button}%
    \usebeamercolor{black}%
    \hspace{3em}%
    \huge{\insertframenumber/\inserttotalframenumber}
}

\setbeamertemplate{footline}[text line]{%
  \parbox{\linewidth}{\vspace*{-8pt}\today\hfill\insertauthor\hfill\insertframenumber/\inserttotalframenumber}}
\setbeamertemplate{navigation symbols}{}

\setbeamercolor{palette primary}{bg=UPblau,fg=white}
\setbeamercolor{palette secondary}{bg=UPblau,fg=white}
\setbeamercolor{palette tertiary}{bg=UPblau,fg=white}
\setbeamercolor{palette quaternary}{bg=UPblau,fg=white}
\setbeamercolor{structure}{fg=UPblau} % itemize, enumerate, etc
\setbeamercolor{section in toc}{fg=UPblau} % TOC sections

\begin{document}
\maketitle

\begin{frame}[t]
    \frametitle{Gliederung}
    \hypersetup{linkcolor=black}
    \tableofcontents%[subsubsectionstyle=hide]
\end{frame}

\section{Definition Datenbank}

\begin{frame}[t]
    \frametitle{\secname}
    \begin{itemize}
        \item organisierte Sammlung von strukturierten Informationen oder Daten
        \item wird typischerweise elektronisch in einem Computersstem gespeichert
        \item Steuerung durch Datenbankmanagementsystem
    \end{itemize}
\end{frame}

\begin{frame}[t]
    \frametitle{\secname}
    \begin{itemize}
        \item Aufgabe: Speicherung von Daten effizient, wiederspruchsfrei und persistent
        \item Unterschied zu Tabellenkalkulationen wie Excel oder ähnliches:
        \begin{itemize}
            \item Speicherung und Bearbeitung von Daten
            \item Personen, die auf Daten zugreifen können
            \item Datenmenge, die gespeichert werden kann
        \end{itemize}
    \end{itemize}
\end{frame}

\section{Typen von Datenbanken}

\subsection{Hierarchische Datenbanken}
\begin{frame}[t]
    \frametitle{\subsecname}
    \begin{itemize}
        \item Datenobjekte stehen in Eltern-Kind-Beziehung zueinander
        \item Beispiel: Dateisystem auf dem PC
    \end{itemize}
\end{frame}

\subsection{Verteilte Datenbanken}
\begin{frame}[t]
    \frametitle{\subsecname}
    \begin{itemize}
        \item bestehend aus mindestens 2 Dateien
        \item Dateien sind an unterschiedlichen Standorten
        \item Dateien können auf veschiedenen Rechnern im selben oder unterschiedlichen Netzwerken gespeichert werden
        \item hat Vorteil das keine Engpässe durch Zugriff auf einen Server entstehen, höhere Ausfallsicherheit
    \end{itemize}
\end{frame}

\subsection{Objektorientierte Datenbanken}
\begin{frame}[t]
    \frametitle{\subsecname}
    \begin{itemize}
        \item Darstellung durch Objekte
        \item Beispiel: C\# Projekt in der Schule 
    \end{itemize}
\end{frame}

\subsection{NoSQL Datenbanken}
\begin{frame}[t]
    \frametitle{\subsecname}
    \begin{itemize}
        \item Speicherung unstrukturierter und semistrukturierter Daten
        \item uneindeutige Definition, wie Daten in Datenbank eingegeben werden
        \item im austeigendem Trend, weil Webanwendungen immer häufiger ud komplexer geworden sind
        \item Redis Key-Value-Store
    \end{itemize}
\end{frame}

\subsection{Document-/ JSON Datenbanken}
\begin{frame}[t]
    \frametitle{\subsecname}
    \begin{itemize}
        \item Speichern, Abrufen und Verwalten dokumentenorientierter Informationen
        \item Speicherung von Daten im JSON Format
        \item Mongo DB
    \end{itemize}
\end{frame}

\subsection{Selbstverwaltende Datenbanken}
\begin{frame}[t]
    \frametitle{\subsecname}
    \begin{itemize}
        \item cloudbasiert
        \item verwenden Machine Learning um Datenbankoptimierung, Sicherheit, Backups, Updates und andere Routineverwaltungsaufgaben zu automatisieren
        \item Übernehmen Aufgaben, die traditionell von Datenbankadministratoren übernommen werden
    \end{itemize}
\end{frame}

\subsection{Relationale Datenbanken}
\begin{frame}[t]
    \frametitle{\subsecname}
    \begin{itemize}
        \item Elemente sind als Satz von Tabellen mit Spalten und Zeilen organisiert
        \item Tabellen stehen in Relationen zueinander
        \item Datensätze sind Identifizierbar über Primär- und Fremdschlüssel
    \end{itemize}
\end{frame}

\subsubsection{Darstellung im ERM Modell}
\begin{frame}[t]
    \frametitle{\subsecname}
    \framesubtitle{\subsubsecname}
    \begin{itemize}
        \item Begriffe: Entität, Kadinalität, Attribut, Beziehung
    \end{itemize}
    \resizebox{\textwidth}{!}{
    \begin{tikzpicture}[auto,node distance=1.5cm]
        % Create an entity with ID node1, label "Fancy Node 1".
        % Default for children (ie. attributes) is to be a tree "growing up"
        % and having a distance of 3cm.
        %
        % 2 of these attributes do so, the 3rd's positioning is overridden.
        \node[entity] (node1) {\scriptsize Entität 1}
        [grow=up,sibling distance=3cm]
        child {node[attribute] {\scriptsize Attribut 1}}
        child {node[attribute] {\scriptsize Attribut 2}}
        child[grow=left,level distance=3cm] {node[attribute] {\scriptsize Attribut 3}};
        % Now place a relation (ID=rel1)
        \node[relationship] (rel1) [below right = of node1] {\scriptsize Beziehung};
        % Now the 2nd entity (ID=rel2)
        \node[entity] (node2) [above right = of rel1]	{\scriptsize Entität 2};
        % Draw an edge between rel1 and node1; rel1 and node2
        \path (rel1) edge node {\scriptsize 1} (node1)
        edge	 node {\scriptsize \(n\)}	(node2);
    \end{tikzpicture}
}
\end{frame}

\begin{frame}[t]
    \frametitle{\subsecname}
    \framesubtitle{\subsubsecname}
    \resizebox{\textwidth}{!}{
    \begin{tikzpicture}[auto,node distance=1.5cm]
        \node[entity] (node1) {\scriptsize Album}
        [grow=up,sibling distance=3cm]
            child {node[attribute] {\scriptsize Albumtitel}}
            child {node[attribute] {\scriptsize Erscheinungsjahr}}
        child[grow=left,level distance=3cm] 
            {node[attribute] {\scriptsize \underline{Album\_ID}}};

        % Now place a relation (ID=rel1)
        \node[relationship] (rel1) 
            [right = of node1] {\scriptsize ist von};

        % Now the 2nd entity (ID=rel2)
        \node[entity] (node2) [right = of rel1]	{\scriptsize Künstler}
            child[grow=up,level distance=1.5cm]
                {node[attribute] {\scriptsize \underline{Interpret\_ID}}}
            child[grow=right,level distance=3cm] 
                {node[attribute] {\scriptsize Name}}
            child[grow=down,level distance=1.5cm] 
                {node[attribute] {\scriptsize Gründungsjahr}};

        % Draw an edge between rel1 and node1; rel1 and node2
        \path 
            (rel1)  
            edge node {\scriptsize \(n\)}       (node1)
            edge node {\scriptsize \(1\)}	(node2);

        \node[relationship] (rel2)
            [below = of node1] {\scriptsize hat};

        \node[entity] (node3) [below right = of rel2] {\scriptsize Lied}
            child[grow=left,level distance=3cm]
                {node[attribute] {\scriptsize Track}}
            child[grow=up,level distance=1.5cm]
                {node[attribute] {\scriptsize \underline{Lied\_ID}}}
            child[grow=right,level distance=3cm]
                {node[attribute] {\scriptsize Titel}};

        \path
            (rel2)
            edge node {\scriptsize \(n\)}       (node1)
            edge node {\scriptsize \(m\)}	(node3);
    \end{tikzpicture}
}
\end{frame}

\subsubsection{Normalisierung}
\begin{frame}[t]
    \frametitle{\subsecname}
    \framesubtitle{\subsubsecname}
    \begin{itemize}
        \item Regeln für die Beseitugung von Redundanzen
        \item Ziel ist Konsistenzerhöhung, zur Verhinderung von Anomalien
    \end{itemize}
\end{frame}

\begin{frame}[t]
    \frametitle{\subsecname}
    \framesubtitle{\subsubsecname}
    1. Normalform
    \begin{itemize}
        \item "`Jedes Attribut der Relation muss einen atomaren Wertebereich haben, und die Relation muss frei von Wiederholungsgruppen sein."'
    \end{itemize}
    \pause
    
\begin{table}
    \resizebox{\textwidth}{!}{
    \begin{tabular}{l|l|l|l|l}
        CD\_ID	&Album				                &Gründungsjahr	&Erscheinungsjahr	&Titelliste \\\hline
        4711	&\makecell[l]{Anastacia \\– Not That Kind}	        &1999		    &2000			    &\makecell[l]{1. Not That Kind,\\2. I’m Outta Love, \\3. Cowboys \& Kisses}\\\hline
        4712	&\makecell[l]{Pink Floyd \\– Wish You Were Here}	&1965		    &1975			    &1. Shine On You Crazy Diamond\\\hline
        4713	&\makecell[l]{Anastacia \\– Freak of Nature}	    &1999		    &2001			    &1. Paid my Dues\\\hline
        
    \end{tabular}}
    \caption{ungeordnete Tablelle}
\end{table}
\end{frame}

\begin{frame}[t]
    \frametitle{\subsecname}
    \framesubtitle{\subsubsecname}
    1. Normalform
    \begin{itemize}
        \item "`Jedes Attribut der Relation muss einen atomaren Wertebereich haben, und die Relation muss frei von Wiederholungsgruppen sein."'
    \end{itemize}
    \begin{table}
    \resizebox{\textwidth}{!}{
    \begin{tabular}{l|l|l|l|l}
        CD\_ID	&Album				                &Gründungsjahr	&Erscheinungsjahr	&Titelliste \\\hline
        4711	&\cellcolor{red!25}\makecell[l]{Anastacia \\– Not That Kind}	        &1999		    &2000			    &\cellcolor{red!25}\makecell[l]{1. Not That Kind,\\2. I’m Outta Love, \\3. Cowboys \& Kisses}\\\hline
        4712	&\cellcolor{red!25}\makecell[l]{Pink Floyd \\– Wish You Were Here}	&1965		    &1975			    &1. Shine On You Crazy Diamond\\\hline
        4713	&\cellcolor{red!25}\makecell[l]{Anastacia \\– Freak of Nature}	    &1999		    &2001			    &1. Paid my Dues\\\hline
        
    \end{tabular}}
    \caption{ungeordnete Tabelle}
\end{table}
\end{frame}

\begin{frame}[t]
    \frametitle{\subsecname}
    \framesubtitle{\subsubsecname}
    1. Normalform
    
\begin{table}
    \resizebox{\textwidth}{!}{
    \begin{tabular}{l|l|l|l|l|l|l}
        CD\_ID	&Albumtitel		    &Interpret	&Gründungsjahr	&Erscheinungsjahr	&Track	&Titel\\\hline
        4711	&Not That Kind		&Anastacia	&1999		    &2000			    &1	    &Not That Kind\\\hline
        4711	&Not That Kind		&Anastacia	&1999		    &2000			    &2	    &I’m Outta Love\\\hline
        4711	&Not That Kind		&Anastacia	&1999		    &2000			    &3	    &Cowboys \& Kisses\\\hline
        4712	&Wish You Were Here &Pink Floyd	&1965		    &1975			    &1	    &Shine On You Crazy Diamond\\\hline
        4713	&Freak of Nature	&Anastacia	&1999		    &2001			    &1	    &Paid my Dues\\\hline
    \end{tabular}}
    \caption{1. Normalform}
\end{table}
\end{frame}

\begin{frame}[t]
    \frametitle{\subsecname}
    \framesubtitle{\subsubsecname}
    2. Normalform
    \begin{itemize}
        \item "`Eine Relation ist genau dann in der zweiten Normalform, wenn die erste Normalform vorliegt und kein Nichtprimärattribut (Attribut, das nicht Teil eines Schlüsselkandidaten ist) funktional von einer echten Teilmenge eines Schlüsselkandidaten abhängt."'
    \end{itemize}
    \pause 
    
\begin{table}
    \resizebox{\textwidth}{!}{
    \begin{tabular}{l|l|l|l|l|l|l}
        CD\_ID	&Albumtitel		    &Interpret	&Gründungsjahr	&Erscheinungsjahr	&Track	&Titel\\\hline
        4711	&\cellcolor{red!25}Not That Kind		&\cellcolor{red!25}Anastacia	&\cellcolor{red!25}1999		    &\cellcolor{red!25}2000			    &1	    &Not That Kind\\\hline
        4711	&\cellcolor{red!25}Not That Kind		&\cellcolor{red!25}Anastacia	&\cellcolor{red!25}1999		    &\cellcolor{red!25}2000			    &2	    &I’m Outta Love\\\hline
        4711	&\cellcolor{red!25}Not That Kind		&\cellcolor{red!25}Anastacia	&\cellcolor{red!25}1999		    &\cellcolor{red!25}2000			    &3	    &Cowboys \& Kisses\\\hline
        4712	&Wish You Were Here &Pink Floyd	&1965		    &1975			    &1	    &Shine On You Crazy Diamond\\\hline
        4713	&Freak of Nature	&Anastacia	&1999		    &2001			    &1	    &Paid my Dues\\\hline
    \end{tabular}}
    \caption{1. Normalform}
\end{table}
\end{frame}

\begin{frame}[t]
    \frametitle{\subsecname}
    \framesubtitle{\subsubsecname}
    2. Normalform
    \begin{table}
    \resizebox{\textwidth}{!}{
    \begin{tabular}{l|l|l|l|l}
        CD\_ID	&Albumtitel		        &Interpret	&Gründungsjahr	&Erscheinungsjahr\\\hline
        4711	&Not That Kind		    &Anastacia	&1999		    &2000\\\hline
        4712	&Wish You Were Here 	&Pink Floyd	&1965		    &1975\\\hline
        4713	&Freak of Nature		&Anastacia	&1999		    &2001\\\hline
    \end{tabular}}
    \caption{2. Normalform Tabelle 1}
\end{table}

\begin{table}
    \resizebox{0.35\textwidth}{!}{
    \begin{tabular}{l|l|l}
        CD\_ID	&Track	&Titel\\\hline
        4711	&1	    &Not That Kind\\\hline
        4711	&2	    &I'm Outta Love\\\hline
        4711	&3	    &Cowboys \& Kisses\\\hline
        4712	&1	    &Shine On You Crazy Diamond\\\hline
        4713	&1	    &Paid my Dues\\\hline
    \end{tabular}}
    \caption{2. Normalform Tabelle 2}
\end{table}
\end{frame}

\begin{frame}[t]
    \frametitle{\subsecname}
    \framesubtitle{\subsubsecname}
    3. Normalform
    \begin{itemize}
        \item "`Die dritte Normalform ist genau dann erreicht, wenn sich das Relationenschema in der 2NF befindet, und kein Nichtschlüsselattribut von einem Schlüsselkandidaten transitiv abhängt."'
    \end{itemize}
\end{frame}

\begin{frame}[t]
    \frametitle{\subsecname}
    \framesubtitle{\subsubsecname}
    3. Normalform
    \begin{table}
    \resizebox{\textwidth}{!}{
    \begin{tabular}{l|l|l|l|l}
        CD\_ID	&Albumtitel		        &Interpret	&Gründungsjahr	&Erscheinungsjahr\\\hline
        4711	&Not That Kind		    &\cellcolor{red!25}Anastacia	&\cellcolor{red!25}1999		    &2000\\\hline
        4712	&Wish You Were Here 	&Pink Floyd	&1965		    &1975\\\hline
        4713	&Freak of Nature		&\cellcolor{red!25}Anastacia	&\cellcolor{red!25}1999		    &2001\\\hline
    \end{tabular}}
    \caption{2. Normalform Tabelle 1}
\end{table}

\begin{table}
    \resizebox{0.35\textwidth}{!}{
    \begin{tabular}{l|l|l}
        CD\_ID	&Track	&Titel\\\hline
        4711	&1	    &Not That Kind\\\hline
        4711	&2	    &I'm Outta Love\\\hline
        4711	&3	    &Cowboys \& Kisses\\\hline
        4712	&1	    &Shine On You Crazy Diamond\\\hline
        4713	&1	    &Paid my Dues\\\hline
    \end{tabular}}
    \caption{2. Normalform Tabelle 2}
\end{table}
\end{frame}

\begin{frame}[t]
    \frametitle{\subsecname}
    \framesubtitle{\subsubsecname}
    3. Normalform
    \begin{columns}
    \begin{column}{0.5\textwidth}
        \begin{table}
            \resizebox{\textwidth}{!}{
            \begin{tabular}{l|l|l|l}
                CD\_ID	&Albumtitel		        &Interpret\_ID     &Erscheinungsjahr\\\hline
                4711	&Not That Kind		    &311               &2000\\\hline
                4712	&Wish You Were Here 	&312               &1975\\\hline
                4713	&Freak of Nature		&311               &2001\\\hline
            \end{tabular}}
            \caption{3. Normalform Tabelle 1}
        \end{table}
    \end{column}
    \begin{column}{0.5\textwidth}
        \begin{table}
            \resizebox{\textwidth}{!}{
            \begin{tabular}{l|l|l}
                Interpret\_ID	&Interpret	&Gründungsjahr\\\hline
                311		        &Anastacia	&1999\\\hline
                312		        &Pink Floyd	&1965\\\hline
            \end{tabular}}
            \caption{3. Normalform Tabelle 2}
        \end{table}
    \end{column}
\end{columns}

\begin{table}
    \resizebox{0.35\textwidth}{!}{
    \begin{tabular}{l|l|l}
        CD\_ID	&Track	&Titel\\\hline
        4711	&1	    &Not That Kind\\\hline
        4711	&2	    &I'm Outta Love\\\hline
        4711	&3	    &Cowboys \& Kisses\\\hline
        4712	&1	    &Shine On You Crazy Diamond\\\hline
        4713	&1	    &Paid my Dues\\\hline
    \end{tabular}}
    \caption{3. Normalform Tabelle 3}
\end{table}
\end{frame}

\section{DBMS}
\begin{frame}[t]
    \frametitle{Datenbankmanagementsysteme}
    \begin{itemize}
        \item Organisation und Strukturierung der Daten
        \item Kontrolle von lesenden und schreibenden Zugriffen auf die Datenbasis
        \item im Unternehmen SQLYog für Relationale Datenbanken verwendet
    \end{itemize}
\end{frame}

\end{document}